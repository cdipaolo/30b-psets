\documentclass[12pt,letterpaper]{hmcpset}
\usepackage[margin=1in]{geometry}
\usepackage{graphicx}

% info for header block in upper right hand corner
\name{}
\class{Math 30b - Orrison}
\assignment{Induction}
\duedate{Friday, September 4}

\begin{document}

\problemlist{2.\{1.ii, 2.ii, 10, 20, 26\}}

% 1.ii %
\begin{problem}[2.1.ii]
    Prove the following formula by induction:

    \[1^3+...+n^3 = \left( 1+...+n \right)^2\]
\end{problem}

\begin{solution}

\end{solution}

% 2.ii %
\begin{problem}[2.2.ii]
    Find a formula for:

    \[\sum_{i=1}^n \left( 2i-1 \right)^2 = 1^2 + 3^2 + 5^2 + ... + \left( 2n-1 \right)^2\]
\end{problem}

\begin{solution}

\end{solution}

% 10 %
\begin{problem}[2.10]
   Prove the Principal of Mathematical Induction from the Well Ordering Principal 
\end{problem}

\begin{solution}

\end{solution}

% 20 %
\begin{problem}[2.20]
    The Fibonacci sequence \(a_1,a_2,a_3,...\) is defined as follows:

    \begin{align*}
        a_1 &= 1\\
        a_2 &= 1\\
        a_n &= a_{n-1} + a_{n-2} \quad \text{for } n \ge 3
    \end{align*}

    (some irrelevant history mentioned here...)\\

    Prove that:

    \[ a_n = \frac{\left(\frac{1+\sqrt{5}}{2}\right)^n - \left(\frac{1-\sqrt{5}}{2}\right)^n}{\sqrt{5}} \]
\end{problem}

\begin{solution}

\end{solution}

% 26 %
\begin{problem}[2.26]
    There is a puzzle consisting of three spindles, with \(n\) concentric rings of decreasing diameter stacked on the first (Figure 1). A ring at the top of a stack may be moved from one spindle to another spindle, provided that it is not placed on top of a smaller ring. For example, if the smallest ring is moved to spindle 2 and the next-smallest ring is moved to spindle 3, then the smallest ring may be moved to spindle 3 also, on top of the next-smallest. Prove that the entire stack of \(n\) rings can be moved onto spindle 3 in \(2^n-1\) moves, and that this cannot be done in fewer than \(2^n-1\) moves.
\end{problem}

\begin{solution}

\end{solution}

\end{document}
