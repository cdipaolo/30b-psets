\documentclass[12pt,letterpaper]{hmcpset}
\usepackage[margin=1in]{geometry}
\usepackage{graphicx}
\usepackage{amsmath}

% CREDS to Jingnan Shi for typing this one %

% info for header block in upper right hand corner
\name{}
\class{Math 30b – Orrison}
\assignment{Sequences and Series}
\duedate{Tuesday, October 6, 2015}

\newcommand{\pn}[1]{\left( #1 \right)}
\newcommand{\abs}[1]{\left| #1 \right|}
\newcommand{\bk}[1]{\left[ #1 \right]}

\newcommand{\fx}{f \left( x \right) =}
\newcommand*\LH{\ensuremath{\overset{\kern2pt L'H}{=}}}

\begin{document}

\problemlist{(Chp. 22) 1(ii, iv, vi, viii), 2(i, iii, v), 4(a). (Chp. 23) 1(ii, iv, vi, viii), 6(a, b).}

% 18.1.ii %
\begin{problem}[22.1.ii]
Verify the limit:

\[ \lim_{n \to \infty} \dfrac{n+3}{n^3+4} = 0\]

\end{problem}

\begin{solution}
\end{solution}

% 22.1.iv %
\begin{problem}[22.1.iv]
Verify the limit:

\[ \lim_{n \to \infty} \dfrac{n!}{n^n} = 0\]

\end{problem}

\begin{solution}

\end{solution}

% 22.1.vi %
\begin{problem}[22.1.vi]

Verify the limit:

\[ \lim_{n \to \infty} \sqrt[n]{n} = 1\]

\end{problem}

\begin{solution}

\end{solution}

% 22.1.x %
\begin{problem}[22.1.viii]

Verify the limit:

\[ \lim_{n \to \infty} \sqrt[n]{a^n+b^n} = \max \left( a, b \right)\text{, }a,b \ge 0. \]

\end{problem}

\begin{solution}
\end{solution}

% 22.5.i %
\begin{problem}[22.2.i]

Find the limit:

\[ \lim_{n \to \infty} \dfrac{n}{n+1} - \dfrac{n+1}{n} \]

\end{problem}

\begin{solution}

\end{solution}

% 22.2.iii %
\begin{problem}[22.2.iii]

Find the limit:

\[ \lim_{n \to \infty} \dfrac{2^n+(-1)^n}{2^{n+1}+(-1)^{n+1}} \]

\end{problem}

\begin{solution}
	
\end{solution}

% 22.2.v %
\begin{problem}[22.2.v]

Find the limit:

\[ \lim_{n \to \infty} \dfrac{a^n-b^n}{a^n+b^n}\]

\end{problem}

\begin{solution}
\end{solution}

% 22.4.a %
\begin{problem}[22.4.a]

Prove that if a subsequence of a Cauchy sequence converges, then so does the original Cauchy sequence.

\end{problem}
\begin{solution}
\end{solution}

% 23.1.ii %
\begin{problem}[23.1.ii]

Decide whether the following infinite series is convergent or divergent:

\[ 1 - \dfrac{1}{3} + \dfrac{1}{5} - \dfrac{1}{7} + \ldots \]

\end{problem}

\begin{solution}

\end{solution}

% 23.1.iv %
\begin{problem}[23.1.iv]

Decide whether the following infinite series is convergent or divergent:

\[ \sum_{n=1}^{\infty} (-1)^n \dfrac{\log{n}}{n}\]

\end{problem}

\begin{solution}

\end{solution}

% 23.1.vi %
\begin{problem}[23.1.vi]

Decide whether the following infinite series is convergent or divergent:

\[ \sum_{n=1}^{\infty} \dfrac{1}{\sqrt[3]{n^2+1}}\]

\end{problem}

\begin{solution}
\end{solution}

% 23.1.viii %
\begin{problem}[23.1.viii]

Decide whether the following infinite series is convergent or divergent:

\[ \sum_{n=1}^{\infty} \dfrac{\log{n}}{n}\]

\end{problem}

\begin{solution}
\end{solution}

% 23.6.a %
\begin{problem}[23.6.a]

Let $f$ be a continuous function on an interval around 0, and let $a_n = f(1/n)$ (for large enough $n$).

Prove that if $\sum_{n=1}^{\infty} a_n$ converges, then $f(0) = 0$.

\end{problem}

\begin{solution}

\end{solution}

% 23.6.b %
\begin{problem}[23.6.b]

Let $f$ be a continuous function on an interval around 0, and let $a_n = f(1/n)$ (for large enough $n$).

Prove that if $f'(0)$ exists and $\sum_{n=1}^{\infty} a_n$ converges, then $f'(0) = 0$.


\end{problem}

\begin{solution}

\end{solution}


\end{document}
